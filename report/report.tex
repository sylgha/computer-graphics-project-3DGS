\documentclass[UTF8,a4paper]{ctexart}

\usepackage[margin=1in]{geometry}
\usepackage{amsmath}
\usepackage{amssymb}
\usepackage{graphicx}
\usepackage{booktabs}
\usepackage{hyperref}
\usepackage{xcolor}
\usepackage{listings}
\usepackage{caption}

\hypersetup{
	colorlinks=true,
	linkcolor=blue,
	urlcolor=blue,
	citecolor=blue
}

\lstset{
	basicstyle=\ttfamily\small,
	keywordstyle=\color{blue!70!black},
	commentstyle=\color{gray!70},
	stringstyle=\color{green!40!black},
	numbers=left,
	numberstyle=\tiny\color{gray},
	stepnumber=1,
	numbersep=6pt,
	frame=single,
	breaklines=true,
	breakatwhitespace=true,
	showstringspaces=false,
	tabsize=2
}

\title{计算机图形学报告\\基于 Gaussian Surfels 的 3DGS 表面重建与可视化实验}
\author{09023301\quad 张凯轩}
\date{2026年1月7日}

\begin{document}
\maketitle

\section{实验目的与任务概述}

本课程项目以 3DGS 为基础,阅读并尝试运行 2024 年 SIGGRAPH 论文 ``High-quality Surface Reconstruction using Gaussian Surfels'' 提出的 Gaussian Surfels 模型。
根据项目要求和个人理解,我对本实验的目标与任务设置为:

\begin{enumerate}
	\item 环境配置与编译:完成 Conda 环境搭建,包含 CUDA 光栅化模块与 SIBR 可视化等组件部分;
	\item 代码阅读与理解:解释 3DGS 的关键训练/渲染流程,梳理 Gaussian Surfels 相比原 3DGS 的核心的改进;
	\item 数据集验证与测试:首先在标准 DTU 场景上训练与测试,检查重建效果与论文描述是否一致;
    随后在 HOPE数据集(桌面物体、存在遮挡)上训练,观察鲁棒性与新视角合成效果
	\item surfels论文改进使用的损失函数基于法线和深度一致性,类似一种自监督的损失函数,
    如果我们尝试引入带有一定先验信息的深度值来进行约束,测试训练效果会不会有所提升;
    实验中可以尝试引入单目深度估计(MDE)模型作为深度先验,
    增加单一深度信息的损失项并重新训练,对比观察一下训练结果。
\end{enumerate}

本项目主要基于官方 Gaussian Surfels 实现进行实验与改进,相关链接如下:

Gaussian Surfels 项目主页:

\url{https://turandai.github.io/projects/gaussian_surfels/}

DTU相关格式整理数据集(包含法线信息):

\url{https://huggingface.co/datasets/turandai/gaussian-surfels-dtu}

自测相关数据集HOPE数据集:

\url{https://github.com/swtyree/hope-dataset}

\section{实验环境与配置}

\subsection{硬件与系统环境}

本项目的训练与可视化均在云服务器上进行,一些性能配置信息如下,可以结合此对实际的训练时间做参考

\begin{table}[h]
\centering
\begin{tabular}{ll}
\toprule
项目 & 配置 \\
\midrule
操作系统 & Ubuntu 22.04 \\
Conda & Miniconda(conda3)\\
Python & 3.7 \\
CUDA & 11.6 \\
GPU & vGPU-32GB(32GB)$\times 1$ \\
CPU & 12 vCPU Intel(R) Xeon(R) Platinum 8375C CPU @ 2.90GHz \\
内存 & 72GB \\
\bottomrule
\end{tabular}
\caption{实验环境}
\end{table}

\subsection{环境创建与编译}
Conda 环境创建:
由于云服务器上直接condea安装整个环境sovling速度较慢,实际配置使用pip安装必要的库步步调试下载,具体不再赘述。

重新编译 CUDA 光栅化模块:
\begin{lstlisting}[language=bash]
cd gaussian_surfels/submodules/diff-gaussian-rasterization
python setup.py install
pip install .
\end{lstlisting}

SIBR 浏览器的编译与安装依赖 CMake,结合调试及报错信息补充安装必要的库和依赖。

如果想要使用相关环境,可以直接使用课程实验GitHub仓库中或者网盘中的环境压缩包进行解压使用,包含训练以及SIBR等所有必要组件。


\section{关键代码与分析}

本节按照数据流动的逻辑,从 Surfel 表示构建、可微渲染、损失优化到最终表面重建,逐步梳理 Gaussian Surfels 的核心技术环节。代码路径均在本仓库中可直接定位。

\subsection{方法总览}

图\ref{fig:overview} 给出了论文的pipeline。左侧为 Gaussian Surfels 表示与初始化(从点云到 surfel),中部对应损失函数,右侧则是 volumetric cutting 与 mesh 重建步骤。

\begin{figure}[h]
	\centering
	\includegraphics[width=0.8\linewidth]{pipeline.png}
	\caption{Gaussian Surfels 训练-重建全流程:左侧为 Surfels 表示与初始化,中间为三类损失,右侧为 volumetric cutting 与 mesh}
	\label{fig:overview}
\end{figure}

\subsection{Gaussian Surfels 表示与初始化}

在经典 3DGS 中,高斯以位置、协方差(缩放+旋转)、颜色(SH 系数)与不透明度表示。
Gaussian Surfels 的核心工作是强调使每个高斯的趋近于切平面:通过旋转参数生成法向,
并在初始化时对某一轴缩放,使其更像面片。

\paragraph{法向表示与提取}
Gaussian Surfels 将每个高斯的旋转矩阵 $\mathbf{R}_i$ 的第三列定义为该 surfel 的法向:
$$
\mathbf{n}_i = \mathbf{R}_i \mathbf{e}_3 = \begin{bmatrix} r_{13} \\ r_{23} \\ r_{33} \end{bmatrix}
$$
其中 $\mathbf{e}_3 = [0, 0, 1]^T$ 为局部坐标系的 z 轴。旋转参数既控制高斯的朝向,也可以表示表面法向。

\paragraph{Surfel 初始化:各向异性压扁}
为使高斯更像薄片状表面元素,初始化时对 z 方向缩放施加极小值:
$$
\mathbf{S}_i = \text{diag}(s_x, s_y, s_z - 10^{10})
$$
使协方差矩阵在法向方向上几乎退化,将体积高斯压扁为近二维的 surfel。同时,若点云提供初始法向 $\mathbf{n}_0$,通过 \texttt{normal2rotation} 将其转换为旋转四元数,确保 surfel 法向与表面法向对齐。

对应实现在 \texttt{gaussian\_surfels/scene/gaussian\_model.py}:\texttt{get\_normal} 函数通过四元数旋转矩阵的第三列提取法向;\texttt{create\_from\_pcd} 函数在启用 surface 配置时,
使用点云法向初始化旋转,并通过 \texttt{scales[..., -1] -= 1e10} 压扁 z 方向尺度。

初始化旋转与 z 缩放:
\begin{lstlisting}[language=Python]
if self.config[0] > 0:
		rots = normal2rotation(torch.from_numpy(normals).to(torch.float32)).to("cuda")
		scales[..., -1] -= 1e10  # squeeze z scaling
else:
		rots = torch.zeros((fused_point_cloud.shape[0], 4), device="cuda")
		rots[:, 0] = 1
\end{lstlisting}

这样的高斯表示可以匹配后续的表面约束损失。

\paragraph{深度与法向的计算}
初始化后,系统在渲染过程中需要精确计算每个像素的深度与法向,以支持表面约束。Gaussian Surfels 获得更可靠的深度图。如图\ref{fig:depthcalc} 所示,该方法先在图像空间中统一视线方向,再将前景高斯投影到深度平面,从而与 surfel 表示相匹配。

基于渲染的深度图 $D(u,v)$(像素 $(u,v)$ 处的深度值),通过反投影可将像素恢复为世界坐标 3D 点:
\begin{equation}
\mathbf{X}(u,v) = D(u,v) \mathbf{K}^{-1} \begin{bmatrix} u \\ v \\ 1 \end{bmatrix}
\end{equation}
其中 $\mathbf{K} = \begin{bmatrix} f_x & 0 & c_x \\ 0 & f_y & c_y \\ 0 & 0 & 1 \end{bmatrix}$ 为相机内参矩阵($f_x, f_y$ 为焦距,$c_x, c_y$ 为主点)。

随后通过有限差分计算深度图的梯度,推导每个像素处的曲面法向:
\begin{equation}
\frac{\partial \mathbf{X}}{\partial u} = D(u,v) \mathbf{K}^{-1} \begin{bmatrix} 1 \\ 0 \\ 0 \end{bmatrix} + \frac{\partial D}{\partial u} \mathbf{K}^{-1} \begin{bmatrix} u \\ v \\ 1 \end{bmatrix}
\end{equation}
\begin{equation}
\frac{\partial \mathbf{X}}{\partial v} = D(u,v) \mathbf{K}^{-1} \begin{bmatrix} 0 \\ 1 \\ 0 \end{bmatrix} + \frac{\partial D}{\partial v} \mathbf{K}^{-1} \begin{bmatrix} u \\ v \\ 1 \end{bmatrix}
\end{equation}

法向由偏导数叉积的归一化得到:
\begin{equation}
\mathbf{n}_{\text{depth}} = \text{normalize}\left(\frac{\partial \mathbf{X}}{\partial u} \times \frac{\partial \mathbf{X}}{\partial v}\right)
\end{equation}

\begin{figure}[h]
	\centering
	\includegraphics[width=0.7\linewidth]{深度计算.png}
	\caption{更精确的深度/法向计算示意:统一视线后,再由深度平面匹配 surfel 中心}
	\label{fig:depthcalc}
\end{figure}

以下为该过程的 Python 实现示例(来自 \texttt{gaussian\_surfels/utils/loss\_utils.py}):
\begin{lstlisting}[language=Python]
def depth2normal(depth, mask, viewpoint_cam):
    """从深度图计算法向:通过反投影与有限差分"""
    K = viewpoint_cam.intrinsics  # 内参矩阵
    H, W = depth.shape
    
    # 反投影:将像素坐标 (u,v) 转为相机坐标中的 3D 点
    u, v = torch.meshgrid(torch.arange(W), torch.arange(H), 
                           indexing='ij')
    u, v = u.float(), v.float()
    ones = torch.ones_like(u)
    uv1 = torch.stack([u, v, ones], dim=-1)  # [H, W, 3]
    
    # K^{-1} @ [u, v, 1]
    K_inv = torch.linalg.inv(K)
    points_dir = (uv1 @ K_inv.T).squeeze()  # [H, W, 3]
    
    # 世界坐标 X = D * K^{-1} @ [u, v, 1]
    X = depth.unsqueeze(-1) * points_dir
    
    # 有限差分:Sobel 算子计算梯度
    dX_du = X[:, :-1, :] - X[:, 1:, :]  # 水平梯度
    dX_dv = X[:-1, :, :] - X[1:, :, :]  # 竖直梯度
    
    # 补齐边界,使尺寸对齐
    dX_du = torch.cat([dX_du, dX_du[:, -1:, :]], dim=1)
    dX_dv = torch.cat([dX_dv, dX_dv[-1:, :, :]], dim=0)
    
    # 叉积:法向 = (∂X/∂u) × (∂X/∂v)
    normal = torch.cross(dX_du, dX_dv, dim=-1)
    normal = torch.nn.functional.normalize(normal, dim=-1)
    
    return normal * mask.unsqueeze(-1)
\end{lstlisting}

这个过程无需真实标注的法向数据,而是基于渲染深度信息,指导 surfel 的优化方向。

\subsection{可微渲染}

基于 Surfel 表示,渲染管线仍采用 3DGS 的 $\alpha$-blending 框架,但输出额外包含法向与深度信息用于几何监督。

\paragraph{3DGS 渲染}
3D Gaussian Splatting 将场景表示为一组 3D 高斯函数的集合。每个高斯 $\mathcal{G}_i$ 的参数定义为:位置(均值)$\boldsymbol{\mu}_i \in \mathbb{R}^3$,协方差矩阵 $\boldsymbol{\Sigma}_i = \mathbf{R}_i \mathbf{S}_i \mathbf{S}_i^T \mathbf{R}_i^T$(其中 $\mathbf{R}_i$ 为旋转矩阵由四元数参数化,$\mathbf{S}_i$ 为缩放矩阵),球谐函数系数 $\mathbf{c}_i$ 用于表示视角相关颜色,以及不透明度 $\alpha_i \in [0,1]$。

渲染时,对于像素 $\mathbf{p}$,颜色由沿视线的高斯按深度排序后进行 $\alpha$-blending,其数学表达为:
\begin{equation}
C(\mathbf{p}) = \sum_{i \in \mathcal{N}} c_i \alpha_i' \prod_{j=1}^{i-1} (1 - \alpha_j')
\end{equation}
其中加权不透明度定义为:
\begin{equation}
\alpha_i' = \alpha_i \exp\left(-\frac{1}{2}(\mathbf{x}_i - \boldsymbol{\mu}_i)^T \boldsymbol{\Sigma}_i^{-1} (\mathbf{x}_i - \boldsymbol{\mu}_i)\right)
\end{equation}
而颜色 $c_i$ 由球谐函数根据视角方向计算得到。

\paragraph{渲染输出与监督}
Gaussian Surfels 在渲染时不仅输出 RGB 图像,还同时输出法向图、深度图与不透明度图,用于几何监督,为后续的表面一致性约束提供了基础:

\begin{lstlisting}[language=Python]
render_pkg = render(viewpoint_cam, gaussians, pipe, background, patch_size)
image, normal, depth, opac = (
		render_pkg["render"], render_pkg["normal"],
		render_pkg["depth"],  render_pkg["opac"]
)
\end{lstlisting}

\subsection{损失函数与监督}

损失函数实现主要位于 \texttt{gaussian\_surfels/utils/loss\_utils.py} 与 \texttt{gaussian\_surfels/train.py}。完整损失函数为多项加权和:
\begin{equation}
\mathcal{L} = \mathcal{L}_{\text{RGB}} + \lambda_{\text{mask}} \mathcal{L}_{\text{mask}} + \lambda_{\text{surf}}(t) \mathcal{L}_{\text{surf}} + \lambda_{\text{opac}} \mathcal{L}_{\text{opac}} + \lambda_{\text{curv}} \mathcal{L}_{\text{curv}}
\end{equation}
其中表面约束权重随训练逐步增强,定义为 $\lambda_{\text{surf}}(t) = 0.01 + 0.1 \cdot \min(2t/T, 1)$ 。

\paragraph{(1) 颜色重建:L1 + DSSIM}
颜色监督采用 L1 范数与结构相似性(SSIM)的加权组合:
\begin{equation}
\mathcal{L}_{\text{RGB}} = (1 - \lambda_{\text{dssim}}) \lVert I - \hat{I}\rVert_1 + \lambda_{\text{dssim}} (1 - \text{SSIM}(I, \hat{I}))
\end{equation}
其中 $I$ 为真实图像,$\hat{I}$ 为渲染图像。
\begin{lstlisting}[language=Python]
Ll1 = l1_loss(image, gt_image)
loss_rgb = (1.0 - opt.lambda_dssim) * Ll1 + opt.lambda_dssim * (1.0 - ssim(image, gt_image))
\end{lstlisting}

\paragraph{(2) 表面约束:深度-法向一致性}
Gaussian Surfels 的核心创新在于自监督的表面一致性约束。基于第 3.2 节介绍的深度与法向计算方式,我们用余弦距离约束渲染法向 $\mathbf{n}_{\text{render}}$ 与深度推导的法向 $\mathbf{n}_{\text{depth}}$ 一致:
\begin{equation}
\mathcal{L}_{\text{surf}} = 1 - \frac{1}{|\Omega|} \sum_{\mathbf{p} \in \Omega} \mathbf{n}_{\text{render}}(\mathbf{p}) \cdot \mathbf{n}_{\text{depth}}(\mathbf{p})
\end{equation}
其中 $\Omega$ 为有效像素区域(mask)。这种自监督约束无需真实法向标注,即可促使高斯形成连续光滑的表面。图\ref{fig:consistency}展示了 surfel 分布、深度中心位置、法向方向以及深度-法向一致性的三种情况对比,其中只有当三者都正确时才能形成高质量的表面重建。

\begin{figure}[h]
	\centering
	\includegraphics[width=0.8\linewidth]{深度法线一致性损失.png}
	\caption{Surfel 深度-法向一致性对比:(a) 正确情况,(b) 法向错误,(c) 深度错误}
	\label{fig:consistency}
\end{figure}
\begin{lstlisting}[language=Python]
normal = torch.nn.functional.normalize(normal, dim=0) * mask_vis
d2n = depth2normal(depth, mask_vis, viewpoint_cam)
loss_surface = cos_loss(normal, d2n)
\end{lstlisting}

\paragraph{(3) 可选单目先验:法向}
当提供单目法向预测(如 normal prior)且对应权重非零时,训练会额外引入以下监督项:
\begin{equation}
\mathcal{L}_{\text{monoN}} = \lVert\mathbf{n}_{\text{render}} - \mathbf{n}_{\text{mono}}\rVert_2
\end{equation}
这些先验信息主要目的是提升几何稳定性,特别是在纹理弱或存在高光等区域,缓解不确定性。

\paragraph{(4) opacity 与曲率}
不透明度正则项抑制高斯过度透明或堆积漂浮:
\begin{equation}
\mathcal{L}_{\text{opac}} = \sum_i \alpha_i (1 - \alpha_i)
\end{equation}
该项在 $\alpha_i = 0.5$ 时达到最大惩罚,促使高斯向完全透明或完全不透明状态收敛。曲率正则项通过惩罚相邻 surfel 法向差异来平滑表面:
\begin{equation}
\mathcal{L}_{\text{curv}} = \sum_{i,j \in \mathcal{N}(i)} \lVert\mathbf{n}_i - \mathbf{n}_j\rVert_2
\end{equation}

\subsection{训练策略}

训练入口位于 \texttt{gaussian\_surfels/train.py} 的 \texttt{training(...)}。其核心流程是:
\begin{enumerate}
	\item 随机采样一个相机视角;
	\item 调用可微渲染器得到彩色图、法向、深度与不透明度等;
	\item 计算多项损失并反向传播;
	\item 更新高斯参数,同时执行 densify/prune 等调整。
\end{enumerate}

主要损失项的组合:
\begin{lstlisting}[language=Python]
Ll1 = l1_loss(image, gt_image)
loss_rgb = (1.0 - opt.lambda_dssim) * Ll1 + opt.lambda_dssim * (1.0 - ssim(image, gt_image))
loss_surface = cos_loss(normal, d2n)
loss = 1.0 * loss_rgb + 0.1 * loss_mask
loss += (0.01 + 0.1 * min(2 * iteration / opt.iterations, 1)) * loss_surface
\end{lstlisting}

\subsection{Mesh 重建与后处理}

本项目的 mesh 生成在渲染脚本中完成:\texttt{gaussian\_surfels/render.py} 在渲染 train 视角时会对可见区域进行点采样与清理,然后调用 \texttt{poisson\_mesh(...)} 执行 Screened Poisson 重建。

\paragraph{(1) 从训练视角采样点云}
渲染 train 集时,通过深度与法向进行采样并做 occupancy grid prune,聚合得到用于重建的点云。

\paragraph{(2) Screened Poisson 与网格清理}
	\texttt{gaussian\_surfels/utils/general\_utils.py} 中的 \texttt{poisson\_mesh} 关键流程包括:
\begin{enumerate}
	\item Screened Poisson 重建(输出 \texttt{\_plain.ply});
	\item 通过 KNN 估计顶点颜色和到采样点的距离;
	\item 基于阈值剔除离群点/面,并尝试补洞;
	\item Laplacian 平滑,生成 \texttt{\_pruned.ply}。
\end{enumerate}

核心片段:
\begin{lstlisting}[language=Python]
ms.generate_surface_reconstruction_screened_poisson(depth=depth, preclean=True, samplespernode=1.5)
ms.save_current_mesh(path + '_plain.ply')

ms.compute_selection_by_condition_per_vertex(condselect=f"q>{thrsh}")
ms.meshing_remove_selected_vertices()
ms.meshing_close_holes(maxholesize=300)
ms.apply_coord_laplacian_smoothing(stepsmoothnum=3, boundary=True)
ms.save_current_mesh(path + '_pruned.ply')
\end{lstlisting}

本实验 Poisson 重建耗时很长(具体的底层计算我不太清楚),因此本项目仅在少量 DTU 场景的训练结果上进行了 mesh 测试;HOPE 场景也仅进行了训练与渲染展示。

图\ref{fig:alpha_cut} 用来解释 volumetric cutting :alpha 混合导致的前景/背景交叉会直接体现在深度曲线上,通过 cut 操作可以去除漂浮的错误高斯并获得干净的深度带入 Poisson。

\begin{figure}[h]
	\centering
	\includegraphics[width=0.6\linewidth]{alpha混合可能导致的深度问题.png}
	\caption{Volumetric cutting:处理 $\alpha$ 混合导致的深度错误}
	\label{fig:alpha_cut}
\end{figure}

\section{改动与结果分析}

\subsection{改动}

SIBR 可视化与相关输出精细化;

\textbf{引入深度先验损失}:原始 Gaussian Surfels 仅包含法向先验监督,本项目额外引入了单目深度先验(如 DA3 depth)的 L1 损失项:
	\begin{equation}
	\mathcal{L}_{\text{depth}} = \lVert D_{\text{render}} - D_{\text{mono}}\rVert_1
	\end{equation}
    需要注意的是,源代码中作者也提供了使用深度先验约束的注释,作者在工作中应该也可能尝试了这种方法,但并未在论文中讨论。
	实验中引入该损失项进行了少部分场景的训练测试,结果并未显著提升,具体分析见后文。



\subsection{实验设置:DTU 与 HOPE 数据集}

\paragraph{DTU}

DTU 数据集选取若干场景进行训练,分别尝试:

全部视角训练;

1:1 half split(通过 \texttt{--idr\_half\_split} 划分 train/test)。


\paragraph{HOPE}

HOPE 数据集选取两个 scene 进行全部训练(子集与 full 版本,子集包含100张图片,full包含全部300张左右帧图像),
并在 SIBR 中进行渲染效果展示。

\subsection{结果与现象分析}

注意,SIBR 可视化的视频均在videos/目录下,可以直接查看,但关于加入了深度先验的训练点云并没有拍摄保存。
此外本节仅统计训练时间与 PSNR 等指标,关于倒角距离参数由于DTU的原始GROUND TRUTH没有获取,且mesh计算时间很长,故没有进行计算。

\subsubsection{DTU:全部训练 vs 1:1 half split}

\paragraph{场景与训练方式}
DTU 部分训练了以下场景:\texttt{scan105, scan106, scan110, scan114, scan118},并对其中部分场景进行了 half split(\texttt{--idr\_half\_split}):\texttt{scan105\_half, scan106\_half, scan110\_half, scan114\_half, scan118\_half}。

\paragraph{训练时间与指标}
训练耗时与质量指标统计如表 \ref{tab:dtu_results}。

\begin{table}[h]
\centering
\begin{tabular}{lcccc}
\toprule
场景 & 模式 & 训练时间(分钟) & 训练 PSNR & 测试 PSNR \\
\midrule
scan105 & 全部 & 22.14 & 34.48 & NaN \\
scan105\_half & half split & 22.67 & 34.20 & 30.31 \\
scan106 & 全部 & 22.71 & 35.80 & NaN \\
scan106\_half & half split & 23.98 & 35.42 & 31.61 \\
scan110 & 全部 & 24.02 & 32.91 & NaN \\
scan110\_half & half split & 24.70 & 32.52 & 29.96 \\
scan114 & 全部 & 24.02 & 31.75 & NaN \\
scan114\_half & half split & 24.47 & 31.54 & 27.02 \\
scan118 & 全部 & 24.16 & 35.62 & NaN \\
scan118\_half & half split & 15.55 & 35.36 & 31.88 \\
\bottomrule
\end{tabular}
\caption{DTU 场景训练结果统计(全部视角训练 vs 1:1 half split)}
\label{tab:dtu_results}
\end{table}

\paragraph{现象分析}
综合表 \ref{tab:dtu_results} 的结果,以及论文中原作者的分析:实验中实际的训练时常
相对作者报告的有增加,可能与硬件差异及环境配置有关,具体原因没有深究。但是DTU数据集各个场景
的训练时间相对一致,PSNR指标也在合理范围内,与论文描述30左右基本一致。

\subsubsection{DTU:Mesh 重建与后处理}

由于 Poisson mesh 重建与后处理耗时较长,本项目仅对 3 个 DTU 输出进行了 mesh 测试,且均生成了 \texttt{poisson\_mesh\_8\_plain.ply} 与 \texttt{poisson\_mesh\_8\_pruned.ply}:

\texttt{scan105}(全部训练输出);
\texttt{scan105\_half}(1:1 half split 输出);
\texttt{scan106}(全部训练输出)。

mesh 后处理流程由 \texttt{poisson\_mesh} 完成,详细的效果图在后面部分体现。

\subsubsection{HOPE:训练结果统计}

HOPE 部分训练了两个场景(scene\_0000、scene\_0001),分别包含子集与 full 版本;另外还在 scene\_0000 上训练了引入深度先验的两个变体。表 \ref{tab:hope_results} 统计了相关结果:

\begin{table}[h]
\centering
\begin{tabular}{lccc}
\toprule
场景 & 配置 & 训练时间(分钟) & 训练 PSNR \\
\midrule
hope\_scene\_0000 & baseline & 13.49 & 24.58 \\
hope\_scene\_0000\_full & baseline & 14.87 & 22.62 \\
hope\_scene\_0001 & baseline & 13.35 & 25.34 \\
hope\_scene\_0001\_full & baseline & 13.57 & 21.87 \\
hope\_scene\_0000\_metric & 深度先验 & 8.12 & 23.80 \\
hope\_scene\_0000\_rel & 深度先验 & 8.26 & 23.50 \\
\bottomrule
\end{tabular}
\caption{HOPE 场景训练结果统计}
\label{tab:hope_results}
\end{table}

\paragraph{观察与分析}
表 \ref{tab:hope_results} 显示:
HOPE 的训练 PSNR相比 DTU明显偏低,应该与场景中存在的遮挡、视角稀疏、以及没有设置掩膜前景提取等因素有关;
full 通常较子集难度更高(耗时略长,PSNR 略低),可能原因是引入更多的视角使得训练更复杂不稳定,产生噪声或floaters更多;
深度先验配置在 HOPE 上表现出明显的加速效果,但有可能是因为平台的算力调度有问题,前面提到的训练时间的较大差异也可能是由于某些不确定因素,这里不做具体解释探讨。

\subsubsection{深度先验实验对比}

本项目在三个场景上验证了引入深度先验损失的效果。表 \ref{tab:depth_prior_comparison} 统计了深度先验与对应原始形式训练的对比结果:

\begin{table}[h]
\centering
\begin{tabular}{lcccc}
\toprule
场景 & 配置 & 训练时间(分钟) & 训练 PSNR & PSNR 差值(dB) \\
\midrule
scan105 & baseline & 22.14 & 34.48 & --- \\
scan105\_da3depth & 深度先验 & 22.14 & 34.50 & +0.02 \\
\midrule
hope\_scene\_0000 & baseline & 13.49 & 24.58 & --- \\
hope\_scene\_0000\_metric & 深度先验 & 8.12 & 23.80 & $-$0.78 \\
hope\_scene\_0000\_rel & 深度先验 & 8.26 & 23.50 & $-$1.08 \\
\bottomrule
\end{tabular}
\caption{深度先验实验对比}
\label{tab:depth_prior_comparison}
\end{table}

\paragraph{现象与分析}
表 \ref{tab:depth_prior_comparison} 及相关观察显示:
\begin{enumerate}
	\item \textbf{PSNR 指标}:
	\begin{itemize}
		\item DTU 场景(scan105):PSNR 几乎无差异,表明深度先验影响相较有限;
		\item HOPE 场景:PSNR 下降 ,可能原因包括:
		\begin{itemize}
			\item 单目深度估计本身存在噪声与尺度歧义,当权重过大时可能会导致约束过度;
			\item 纹理丰富、遮挡复杂的真实场景,过强的几何约束可能损害了 RGB 渲染精度。
		\end{itemize}
	\end{itemize}
	\item \textbf{权重调整}:后续工作应尝试类似的动态衰减权重或自适应平衡,以在保持几何质量的同时需要注意维持 RGB 精度。
\end{enumerate}

由于无法计算倒角距离等几何指标,无法直接评估深度先验对几何质量的提升,这里只能使用 PSNR 作为参考。
但从整体结果来看,实验中深度先验对最终的 RGB 重建精度影响有限,这与证明了作者使用的深度-法向一致性约束已经比较有效。
同时也揭示了一个现象,在3D重建任务中,目前的科研工作往往需要在几何质量与渲染精度之间进行权衡,两者很难做到兼顾,这也为后续的科研工作提供了思路与方向。

\subsection{结果截图}
\texttt{screenshots/} 目录中保留了 MeshLab 可视化截图示例,包含不同场景的 mesh 重建结果与模型展示。

\begin{figure}[h]
	\centering
	\includegraphics[width=0.4\linewidth]{../screenshots/bird_mesh103.png}
	\caption{Bird 场景 mesh 重建结果(103)}
\end{figure}

\begin{figure}[h]
	\centering
	\includegraphics[width=0.4\linewidth]{../screenshots/toy_mesh100.png}
	\caption{Toy 场景 mesh 重建结果(100)}
\end{figure}

\begin{figure}[h]
	\centering
	\includegraphics[width=0.4\linewidth]{../screenshots/bird_mesh204.png}
	\caption{Bird 场景 mesh 重建结果(204)}
\end{figure}

\begin{figure}[h]
	\centering
	\includegraphics[width=0.4\linewidth]{../screenshots/toy_mesh201.png}
	\caption{Toy 场景 mesh 重建结果(201)}
\end{figure}

\begin{figure}[h]
	\centering
	\includegraphics[width=0.4\linewidth]{../screenshots/bird_model00.png}
	\caption{Bird 场景模型可视化(00)}
\end{figure}

\begin{figure}[h]
	\centering
	\includegraphics[width=0.4\linewidth]{../screenshots/toy_model103.png}
	\caption{Toy 场景模型可视化(103)}
\end{figure}

\section*{链接}

课程实验GitHub 仓库:\url{https://github.com/sylgha/computer-graphics-project-3DGS}

完整项目文件(包含 data、output等):\url{https://pan.quark.cn/s/f81a73635991},提取码:KHCA


\end{document}
